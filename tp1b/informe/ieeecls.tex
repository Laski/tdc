%======================================================================
%----------------------------------------------------------------------
%               XX                           X
%                                            X
%               XX    XXX   XXX   XXX   XXX  X  XXXX
%                X   X   X X   X X   X X   X X X
%                X   XXXXX XXXXX XXXXX X     X  XXX
%                X   X     X     X     X   X X     X
%               XXX   XXX   XXX   XXX   XXX  X XXXX
%----------------------------------------------------------------------
%  	         SPECIFICATION FOR COMMON IEEE STYLES
%----------------------------------------------------------------------
%               Gregory L. Plett, Istv\'{a}n Koll\'{a}r.
%======================================================================

\documentclass[%
	%draft,
	%submission,
	%compressed,
	final,
	%
	%technote,
	%internal,
	%submitted,
	%inpress,
	%reprint,
	%
	%titlepage,
	%notitlepage,
	%anonymous,
	narroweqnarray,
	inline,
	twoside,
%        invited,
	]{ieee}

\newcommand{\latexiie}{\LaTeX2{\Large$_\varepsilon$}}
\usepackage[utf8]{inputenc}

%\usepackage{ieeetsp}	% if you want the "trans. sig. pro." style
%\usepackage{ieeetc}	% if you want the "trans. comp." style
%\usepackage{ieeeimtc}	% if you want the IMTC conference style

% Use the `endfloat' package to move figures and tables to the end
% of the paper. Useful for `submission' mode.
%\usepackage {endfloat}

% Use the `times' package to use Helvetica and Times-Roman fonts
% instead of the standard Computer Modern fonts. Useful for the 
% IEEE Computer Society transactions.
%\usepackage{times}
% (Note: If you have the commercial package `mathtime,' (from 
% y&y (http://www.yandy.com), it is much better, but the `times' 
% package works too). So, if you have it...
%\usepackage {mathtime}

% for any plug-in code... insert it here. For example, the CDC style...
%\usepackage{ieeecdc}

\begin{document}

%----------------------------------------------------------------------
% Title Information, Abstract and Keywords
%----------------------------------------------------------------------
\title[Traceroute]{%
      Traceroute}

% format author this way for journal articles.
% MAKE SURE THERE ARE NO SPACES BEFORE A \member OR \authorinfo
% COMMAND (this also means `don't break the line before these
% commands).
\author{Axel J. Iglesias, Agustín Martinez Suñe, Nahuel Lascano
}

% format author this way for conference proceedings
%\author[PLETT AND KOLL\'{A}R]{%
        %Gregory L. Plett\member{Student Member},\authorinfo{%
        %Department of Electrical Engineering,\\ 
        %Stanford University, Stanford, CA 94305-9510.\\
        %Phone: $+$1\,650\,723-4769, email: glp@simoon.stanford.edu}%
%\and{}and%
%\and{}Istv\'{a}n Koll\'{a}r\member{Fellow}\authorinfo{%
        %Department of Measurement and Instrument Engineering,\\ 
        %Technical University of Budapest, 1521 Budapest, Hungary.\\
        %Phone: $+$\,36\,1\,463-1774, fax: +\,36\,1\,463-4112, 
        %email: kollar@mmt.bme.hu}
%}

\journal{Trabajo Práctico 1b, 2do cuatrimestre, 2013}
%\titletext{}
%\ieeecopyright{0018--9456/97\$10.00 \copyright\ 1997 IEEE}
%\lognumber{xxxxxxx}
%\pubitemident{S 0018--9456(97)09426--6}
\loginfo{Trabajo Práctico 1b, 2do cuatrimestre de 2013}
\firstpage{1}

%\confplacedate{Ottawa, Canada, May 19--21, 1997}

\maketitle               

\begin{abstract} 
En este trabajo vamos a implementar un traceroute y hacer un análisis de la rutas que realizan los paquetes por la red.\end{abstract}

%\begin{keywords}
%Style file, \latexiie, Microsoft Word, IEEE Publications, Instrumentation
%and Measurement Technology Conference, IMTC.
%\end{keywords}

%----------------------------------------------------------------------
% SECTION I: Introduction
%----------------------------------------------------------------------
\section{Introducción}
El primer paso para desarrollar el TP fue familiarizarse con las herramientas que íbamos a utilizar. Para esto, buscamos distintas implementaciones de traceroute con scapy y analizamos su funcionamiento.

Para los análisis decidimos utilizar 3 Universidades: Rusia (www.msu.ru), China (www.cuhk.edu.hk) y Cuba (www.uh.cu). Para cada caso encontramos interesantes resultados que vamos a explicar en su correspondiente punto.

%----------------------------------------------------------------------
% SECTION II: ej 1
%----------------------------------------------------------------------

\section{Estimación de RTT}
El primer punto fue hacer un análisis teórico del tiempo de respuesta entre las distintas universidades y el host local y compararlo con la implementación nuestra y la del Sistema Operativo.
\subsection{Teórico}
Para la estimación de los RTT's teóricos BLA BLA BLA

Obteniendo como resultado: 
\begin{itemize}
\item Universidad Estatal de Moscú: 13500km / 67.435 ms \\
\item Universidad de la Habana: 6900km / 34.507 ms \\
\item Universidad China de Hong Kong: 18500km / 92.46 ms \\
\end{itemize}
\subsection{Práctico}
Para este análisis implementamos de distintas maneras el traceroute. Sin embargo, varias tenían algunas irregularidades como por ejemplo, que en los laboratorios devolvió resultados muy particulares que más adelante se detallan. La implementación que se encuentra en el TP se "cuelga'' esperando respuestas que nunca llegan, pero se puede continuar arpetando ctrl+c. 

El Sistema Operativo usado fue Mac OS X 10.9.
\subsubsection{Implementación en Scapy}
Como se observa en el código, se utiliza el protocolo TCP para hacer el análisis. La idea es sencilla, enviar los paquetes a destino aumentando el TTL hasta llegar al destino. Por cada paquete, identificamos el RTT, la IP que nos está respondiendo, si es el destino y un cálculo que nos servirá para buscar enlaces transatlánticos.

Por otro lado, se hicieron pruebas en 3 rangos horarios distintos.



Los resultados obtenidos fueron:
\begin{tabbing}
\\ \\

{Hora  }		\={Rusia  } 	\={China  }	\={Cuba  } \\
12Hs \>277ms \>361ms \>No Responde\\
16Hs \>546ms	\>626ms	\>No Responde\\
18Hs \>354ms	\>402ms	\>No Responde

\end{tabbing}

Es interesante ver que a las 19hs de Moscú (16hs de Bs As) es el momento donde mayor RTT hay. Sin embargo haya que tener en cuenta el detalle de que, al ser una comunicación transatlántica, posiblemente el paquete viaje por distintas zonas horarias y esto aumentará o disminuirá el RTT dependiendo del consumo de cada país.

El caso de Cuba nos desorientó bastante. No pudimos conseguir respuesta mediante un ping ni tampoco mediante un traceroute. En algunos casos, todos los paquetes tenían como origen de la respuesta el router local (192.168.0.1). No pudimos descifrar a que se debe este accionar.

\subsubsection{Implementación del Sistema Operativo}
Utilizando el traceroute del SO obtuvimos estos resultados: 
\begin{tabbing}
{Hora  }		\={Rusia  } 	\={China  }	\={Cuba  } \\
12Hs \>270ms \>357ms \>No Responde\\
16Hs \>603ms	\>457ms	\>No Responde\\
18Hs \>325ms	\>521ms	\>No Responde
\end{tabbing}
\subsection{Análisis}
Luego de haber hecho las pruebas, vemos que la implementación del SO no tiene mayores diferencias a la que implementamos nosotros. Aunque si cuenta con más herramientas que las que nosotros hicimos. 
Como mencionamos antes, el horario en el que se realizan las pruebas influye en el rendimiento que obtenemos. Este rendimiento se verá afectado por el horario no solo actual sino también de los países de los enlaces que utilice el paquete para llegar a destino.

FALTA ANALIZAR Y GRAFICAR EL ULTIMO PUNTO DEL 3.1.1


%----------------------------------------------------------------------
% SECTION II: ej 2
%----------------------------------------------------------------------

\section{Enlaces transatlánticos}
En esta segunda parte, vamos a encontrar los enlaces transatlánticos que usa nuestro paquete.
\subsection{Implementación}
Para resolver este punto se modificó el código del traceroute para que vaya acumulando las mediciones y calcule el promedio, el desvío y la posibilidad de que el enlace sea transatlántico.

A su vez, implementaos un script que toma una lista de IP's y utiliza la web propuesta por la materia para buscar su geolocalización.

Utilizando estas dos herramienta, comparamos los resultados teóricos con los verdaderos saltos.

En términos generales, la heurística detectó algunos enlaces transatlánticos, pero para la gran mayoría no lo hizo. Hubo tanto falsos positivos como verdaderos negativos. Creemos que esto puede deberse no tanto al cálculo en si, sino al error general de la implementación que no pudimos resolver.

\subsection{Análisis de geolocalización}
Luego de realizar las pruebas,  nos encontramos con que varias veces la heurística toma como enlace transatlántico al segundo enlace en un nuevo continente. A continuación damos un ejemplo:

\begin{tabbing}
IP	\hspace*{2cm}			 	\=Geolocalización	\=Heurística \\
192.168.0.1	\>Router \>False \\
181.167.38.1	\>Arg	\>False \\
200.89.165.49	\>Arg	\>False \\	
200.89.165.6	\>Arg	\>False \\
200.89.165.150	\>Arg	\>False \\
64.214.130.253	\>EUA	\>False \\
67.17.74.46	\>EUA	\>True\\
146.82.54.2	\>EUA	\>True\\
194.85.40.129	\>Rusia	\>True\\
194.85.40.133	\>Rusia	\>True\\
194.190.254.118	\>Rusia	\>False\\
93.180.0.174	\>Rusia	\>False\\
188.44.33.41	\>Rusia	\>False\\
188.44.33.22	\>Rusia	\>False\\
93.180.0.18	\>Rusia	\>False\\
\end{tabbing}

Esta situación se repite en cada uno de los casos. Creemos que puede deberse a la falta d información de algunos hops.

Por otro lado, aumentar m lo único que provocaría es menos falsos positivos pero no conseguiríamos encontrar los enlaces que de por si no encontramos. Disminuir m nos daría más falsos positivos.

Otra situación que nos llamó la atención en la ubicación de los enlaces transatlánticos. Observemos que para enviar el paquete a Rusia, el mismo debe pasar por EUA. En el caso de Cuba:

\begin{tabbing}
IP	\hspace*{2cm}			 	\=Geolocalización\\
192.168.0.1	\>Argentina \\
181.167.38.1	\>Argentina \\
200.89.165.33	\>Argentina \\
200.89.165.2	\>Argentina \\
200.89.165.101	\>Argentina \\
200.89.165.86	\>Argentina \\
200.89.165.85	\>Argentina \\
200.89.165.1	\>Argentina \\
200.89.165.250	\>Argentina \\
64.209.94.97	\>EUA \\
67.16.139.65	\>EUA \\
67.16.139.65	\>EUA \\
213.140.53.109	\>España \\
176.52.255.6	\>España \\
94.142.118.33	\>España \\
94.142.118.45	\>España \\
200.0.16.73	\>Cuba \\
\end{tabbing}

Nuevamente el recorrido cruza por EUA. Pero para el caso de Hong Kong:

\begin{tabbing}
Hop \=IP	\hspace*{2cm}			 	\=Geolocalización\\
1  \>192.168.0.1     \> Argentina \\
2  \>181.167.38.1    \> Argentina \\
5  \>200.89.165.49   \> Argentina \\
6  \>200.89.165.2    \> Argentina \\
9  \>200.89.165.101  \> Argentina \\
10 \>200.89.165.86   \> Argentina \\
11 \>195.22.220.128  \> Italia \\
12 \>195.22.223.164  \> Italia \\
13 \>195.22.223.142  \> Italia \\
14 \>115.160.187.102 \> Hong Kong \\
15 \>175.45.11.98    \> Hong Kong \\
16 \>203.188.117.34  \> Hong Kong \\
17 \>137.189.192.250 \> Hong Kong \\
18 \>137.189.9.57    \> Hong Kong \\
19 \>137.189.11.73 	\> Hong Kong \\
\end{tabbing}

Con esto llegamos a dos conclusiones. La primera es que en Argentina contamos con enlaces a distintos continentes que se utilizan de maneras diversas. Esto provoca que cruzar a Europa pueda hacerse directamente (Arg-Italia) o bien pasando por America del Norte (Arg-EUA-España).
La segunda, más que una conclusión es un interrogante. Nos llama la atención que particularmente el paquete a Cuba pacer por EUA. Al saber que a nivel continental muchos enlaces no son simples algoritmos de routeo sino que están dirigidos, nos parece un dato importante el que obtuvimos con este análisis.

Por último, otra heurística que se podría llegar a utilizar en cierta medidas, es el análisis de los prefijos de IP. Creemos que podría ser bastante acertada dada la distribución de los números IP globalmente.


\end{document}
