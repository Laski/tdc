Para poder entender el trabajo que hicimos es preciso en primer lugar comprender
el funcionamiento del protocolo ARP. El protocolo se utiliza para averiguar a
qué dirección MAC corresponde una dirección IP determinada dentro de una red. El
nodo que quiere averiguar esta información hace un broadcast Ethernet de un
paquete ARP ``who-has'' con la dirección IP a consultar, y alguno de los nodos
de la red que tenga la respuesta en su tabla ARP le envía un paquete ARP
``is-at'' con la dirección MAC correspondiente.

La herramienta que desarrollamos captura todos los paquetes de tipo ARP que
circulen en la red mediante el comando $sniff()$ del paquete Scapy. Por cada
paquete guardamos diferentes datos: la IP destino (de dispositivo nodo se quiere
averiguar la MAC), la IP origen (qué dispositivo es el que quiere averiguarla),
y el tipo de paquete (who-has o is-at). También podrían guardarse los datos de las MAC Address origen y destino de cada paquete, simplemente analizando la cabecera del Ethernet Frame. Como explicamos más adelante, elegimos las direcciones IP como simbolos y es por eso que no nos percatamos de guardar las MAC Address. De esta manera, la fuente de información
que queda definida es la red visible al nodo desde el que estamos capturando
paquetes, y el conjunto de simbolos de esta fuente son las direcciones IP de los
nodos.

Mediante la captura de los paquetes vamos a poder caracterizar cada nodo de la
red, sabiendo qué nodos se conectan más frecuentemente con qué otros nodos, y
además vamos a poder calcular cuál es la entropía de la red.

Realizamos la captura de paquetes de dos fuentes diferentes. En primer lugar
recolectamos durante una hora los datos de los paquetes de la red Wi-Fi de los
laboratorios de la facultad, y por otro lado recolectamos datos de los paquetes
de una red montada en el lugar de trabajo de un compañero.