A lo largo del TP descubrimos que muchas cosas pueden averiguarse utilizando
herramientas simples y al acceso de cualquiera con conocimientos básicos sobre
redes o programación. Simplemente mirando los paquetes ARP (una mínima fracción
de los totales) se puede descubrir los roles de los nodos, la cantidad promedio
de dispositivos que se conectan por día, los horarios de mayor carga, la carga
promedio del router y del servidor de DHCP, las funciones que cumple la red y
cuánta comunicación hay entre dispositivos por fuera del router.

Todo esta información bien podría ser usada por un administrador de red para
diagnosticar problemas como por un hacker con fines contrarios a los
de la organización dueña de la red. En cualquier caso, el TP nos ayudó a tomar
conciencia sobre la importancia de la seguridad en las redes, y en particular
las públicas, dado que también aprendimos a usar herramientas que pueden obtener
mucha más información de las que nos limitamos a coleccionar en este TP.

En particular, el uso de estas herramientas y la necesidad de sacar conclusiones a partir de las capturas nos hizo entender que no es dificil encontrarse en el día a día con distintos tipos de redes que tienen comportamientos y topologías diferentes. Y que estas diferencias se pueden inferir sin mucha dificultad tan solo a partir de muestras de las capturas del tráfico de los paquetes en la red.